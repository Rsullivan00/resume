%%%%%%%%%%%%%%%%%%%%%%%%%%%%%%%%%%%%%%%%%
% Friggeri Resume/CV
% XeLaTeX Template
% Version 1.2 (3/5/15)
%
% This template has been downloaded from:
% http://www.LaTeXTemplates.com
%
% Original author:
% Adrien Friggeri (adrien@friggeri.net)
% https://github.com/afriggeri/CV
%
% License:
% CC BY-NC-SA 3.0 (http://creativecommons.org/licenses/by-nc-sa/3.0/)
%
% Important notes:
% This template needs to be compiled with XeLaTeX and the bibliography, if used,
% needs to be compiled with biber rather than bibtex.
%
%%%%%%%%%%%%%%%%%%%%%%%%%%%%%%%%%%%%%%%%%

\documentclass[]{friggeri-cv} % Add 'print' as an option into the square bracket to remove colors from this template for printing

\begin{document}

\header{Rick}{Sullivan}{software engineer} % Your name and current job title/field

%----------------------------------------------------------------------------------------
%	SIDEBAR SECTION
%----------------------------------------------------------------------------------------

\begin{aside} % In the aside, each new line forces a line break
\section{contact}
19 Clementina St.
Unit 306
San Francisco, CA 94105
~
+1 (503) 867 6118
~
\href{mailto:rick@ricksullivan.net}{rick@ricksullivan.net}
\href{http://www.ricksullivan.net}{www.ricksullivan.net}
\href{http://github.com/rsullivan00}{GitHub://rsullivan00}
%
\section{languages}
Python, Java, Javascript
CSS3 \& HTML5
%
\section{technologies}
Django, React, jQuery
Git, SCSS, LaTeX, Solr
Ubuntu, Unix, Windows
\end{aside}
%----------------------------------------------------------------------------------------
%	EDUCATION SECTION
%----------------------------------------------------------------------------------------

\section{education}

\begin{entrylist}

%------------------------------------------------
\entry
{2011 -- 2015}
{Bachelor of Science {\normalfont in Computer Science \& Engineering}}
{Santa Clara University}
{GPA: 3.876 -- \emph{magna cum laude}}


%------------------------------------------------

\end{entrylist}

%----------------------------------------------------------------------------------------
%	WORK EXPERIENCE SECTION
%----------------------------------------------------------------------------------------

\section{experience}

\begin{entrylist}

%------------------------------------------------
%\newcommand{\mydot}{$\cdot$}
 
\entry
{2015 -- Now}
{Accenture Technology}
{San Francisco, California}
{\emph{Technology Architecture Analyst} \\
    \textbf{Client A:} Designed and implemented intelligent search using Apache Solr and a Java integration layer. Coordinated functional and technical designs with the appropriate teams, proved the concept, then developed the final solution.
    \vspace{1mm}\\
    \textbf{Client B:} Designed and implemented a Jenkins solution for automating application deployment, accounting for easy configuration and extensibility. Triaged and resolved defects in configuration, build processes, and Java applications.\\
Solr \mydot Jenkins \mydot Bash \mydot PowerShell \mydot Java \mydot Oracle \mydot IBM DB2}
%Detailed achievements:
%\begin{itemize}
%	\item Designed and implemented new Jenkins automation solution
%	\begin{itemize}
%		\item Developed Bash and PowerShell scripts to incorporate domain knowledge into the deployment system
%		\item Improved existing scripts to allow for more flexible use in future releases
%		\item Technologies used: Jenkins, Bash, PowerShell, Batch, Oracle
%	\end{itemize}
%\end{itemize}}

\entry
{2015 -- 2016}
{Finrise, Inc.}
{San Mateo, California}
{\emph{Fullstack Engineer} \\
Created \href{www.finrise.com}{www.finrise.com} in free time outside of my daily consulting work.
Developed both front and back end features for the vet management Django application at \href{www.vetary.com}{www.vetary.com}, iterating quickly as the small company evolved.\\
Python \mydot ReactJS \mydot PostgresSQL \mydot SCSS \mydot Gulp}

\entry
{Summer 2014}
{Accenture Technology}
{San Jose, California}
{\emph{Business \& System Integration Analyst} \\
Contributed to a MEAN stack crowdsourcing application (MongoDB, ExpressJS, AngularJS, NodeJS), following Agile processes. \\
% Designed and created SharePoint onboarding website for new project roll-ons.\\
AngularJS \mydot NodeJS \mydot ExpressJS \mydot MongoDB \mydot SharePoint \mydot Visio}
%Detailed achievements:
%\begin{itemize}
%	\item Contributed to MEAN stack crowdsourcing application (MongoDB, ExpressJS, AngularJS, NodeJS)
%	\begin{itemize}
%		\item Participated in Agile development
%	\end{itemize}
%
%	\item Created SharePoint onboarding website for new project roll-ons
%		\begin{itemize}
%			\item Gathered necessary information from teams across the client
%			\item Designed and developed site, focusing on ease of use
%		\end{itemize}
%
%	\item Documented physical architecture of a large-scale application
%\end{itemize}}

% \entry
% {Summer 2013}
% {InComm Digital}
% {Portland, Oregon}
% {\emph{Software Engineer Intern} \\
% Created a dynamic API documentation website. Developed sample applications on top of API endpoints, including code examples for developers. Learned and participated in Agile development.\\
% jQuery \mydot SCSS \mydot C\# \& .NET \mydot SQL Server \mydot Bootstrap \mydot Agile}
%Detailed achievements:
%\begin{itemize}
%	\item Created documentation website for a new web API for creating and managing gift cards
%	\begin{itemize}
%		\item Used jQuery, ASP.NET, and SCSS for developing and designing the site.
%		\item Learned and participated in Agile development
%	\end{itemize}
%
%	\item Created sample applications on top of API endpoints
%		\begin{itemize}
%			\item Made sample webpages to show capabilities of the API, including code examples
%		\end{itemize}
%\end{itemize}}

%------------------------------------------------

\end{entrylist}

%----------------------------------------------------------------------------------------
%	PROJECTS SECTION
%----------------------------------------------------------------------------------------

\section{projects}

\begin{entrylist}

%------------------------------------------------
\entry
{2015}
{FoodReader}
{\href{https://github.com/Rsullivan00/labelRecognizer}{github.com/Rsullivan00/labelRecognizer}}
{Computer-vision project that extracts nutritional data from images of USDA food labels. Won best-in-session at my senior design conference.}

\entry
{2015}
{SCU Transcript}
{\href{http://www.scutranscript.com}{www.scutranscript.com}}
{Parses transcript text to an intermediate format. Can produce a perfectly-formatted LaTeX PDF file or organized JSON containing the transcript info.}

%------------------------------------------------

\end{entrylist}



%----------------------------------------------------------------------------------------
%	AWARDS SECTION
%----------------------------------------------------------------------------------------

\section{awards}

\begin{entrylist}

%------------------------------------------------
\entry
{2015}
{Senior Design Session Winner}
{Bannan School of Engineering, Santa Clara University}
{Awarded for best capstone project in session, by a panel of industry judges.}

\entry
{2011 -- 2015}
{University Honors College}
{Santa Clara University}
{Graduate of the Santa Clara Honors College.}

\entry
{2011 -- 2015}
{Provost Scholar}
{Bannan School of Engineering, Santa Clara University}
{Half-tuition scholarship for four years.}

%------------------------------------------------

\end{entrylist}

%----------------------------------------------------------------------------------------
%	INTERESTS SECTION
%----------------------------------------------------------------------------------------

\section{interests}

\textbf{professional:} web applications, prototyping, recommender systems, computer vision, NLP\\
\textbf{personal:} music, cooking, bouldering, lifting, craft beer 
%----------------------------------------------------------------------------------------
\end{document}
